\documentclass[letterpaper,11pt,notitlepage]{article}
% define the title
\usepackage{amsmath}
\usepackage{fixltx2e}
\usepackage{graphicx}
\usepackage[labelformat=empty]{caption}
\usepackage{subcaption}
\usepackage{listings}
\usepackage{color}
\usepackage{relsize}
\usepackage{wrapfig}
\usepackage{fontspec}
\usepackage{hyperref}
\usepackage{hanging}
\usepackage{textcomp,    % for \textlangle and \textrangle macros
            xspace}
\newcommand\la{\textlangle}  % set up short-form macros
\newcommand\ra{\textrangle\xspace}
\newcommand\rans{\textrangle}
\lstset{
  language={matlab},
  basicstyle=\ttfamily\smaller\relax,	
  tabsize=4,                              % Default tab size
  showtabs=false,                         % Dont make tabls visible
  columns=flexible,                       % Column formatc
  commentstyle=\color{mygreen}\textit,           % comment style
  extendedchars=true,              % lets you use non-ASCII characters; for 8-bits encodings only, does not work with UTF-8
  numbers=left,                    % where to put the line-numbers; possible values are (none, left, right)
  numbersep=8pt,                   % how far the line-numbers are from the code
  numberstyle=\tiny\color{mygray}, % the style that is used for the line-numbers
  stepnumber=1,
  xleftmargin=2cm,
  breaklines=true,
  keywordstyle=\color{mynavy}
}
\hypersetup{colorlinks=true,linkcolor=blue}
\renewcommand*{\UrlFont}{\ttfamily\smaller\relax}

\addtolength{\topmargin}{-1in}
\addtolength{\textheight}{1.75in}
\setlength\parindent{24pt}

\definecolor{mygreen}{rgb}{0.1020,0.5961,0.3137}
\definecolor{mygray}{rgb}{0.5,0.5,0.5}
\definecolor{light-gray}{rgb}{0.8,0.8,0.8}
\definecolor{mynavy}{rgb}{0.1922,0.2118,0.5843}

\begin{document}

\begin{center}
	Homework 3 - Feature Engineering\\
	2015 Spring, Machine Learning\\
	Choong-Wan Woo\\
    Kaggle Username: CU\_chwo9116\\
	\today\\
\end{center}

\hspace*{-1cm}\textbf{Explanation.}  \rule{10.5cm}{0.4pt}\\

\noindent 1) I removed any accent symbols using the \textbf{strip\_accents} option of the CountVectorizer.\\  
\noindent 2) I removed numbers and punctuations using the \textbf{preprocessor} option of the CountVectorizer.\\
\noindent 3) Using the \textbf{tokenizer} option and WordNetLemmatizer provided by the nltk package, I did a lemmatization.\\
\noindent 4) I removed english stopwords using the \textbf{stop\_words} option of the CountVectorizer.\\
\noindent 5) I created an additional feature that consists of two words in a row in the document using the \textbf{ngram\_range} option of the CountVectorizer.\\

\end{document}